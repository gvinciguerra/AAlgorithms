\section{Family of uniform hash functions}

The notion of pairwise independence says that, for any $x_1 \neq x_2, c_1, c_2
\in \mathbb{Z}_p$, we have that
\begin{equation}
  \label{equation:pairwise-independence}
  \prob{h(x_1)= c_1, h(x_2) = c_2} = \prob{h(x_1) = c_1} \times \prob{h(x_2) = c_2}
\end{equation}
In other words, the joint probability is the product of the two individual probabilities.
Show that the family of hash functions $\mathcal{H} = \{h_{a,b}(x) = ((ax + b) \bmod p) \bmod m :
a \in \mathbb{Z}^*_p,
b \in \mathbb{Z}_p \}$ is \emph{pairwise independent}
where $p$ is a sufficiently large prime number ($m + 1 \leq p \leq 2m$).

\subsection{First proof}

By linear algebra,~\label{modulo_cycle}$a + (kp) \bmod p = c \forall k \in \mathbb{N}$.
If we were to cap $a + pk \bmod p = c \forall k \in K, K_{N} = {k_{i}: k_{i} < N}$
\begin{equation*}
  \prob{h(x_{i}) = c_{i}} = \frac{1}{m^{2}} = \prob{h(x_{1}) = c_{1}, h(x_{2}) = c_{2}}
\end{equation*}
Given $x_{i}, d$, we define as $m_{i} = (a x_{i} + b)$; since $(a x_{i} + b)$
is a \emph{linear transformation} $m_{i}$ is unique.
It follows trivially that there are~\label{p_over_m}$\frac{p}{m}$ values for which $m_{i} \bmod p = d$
since $p > m$ and by~\ref{modulo_cycle}.
The same goes for $x_{1}, x_{2}$:
\begin{gather*}
  (a x_{1} + b) \bmod p = d \\
  (a x_{2} + b) \bmod p = e
\end{gather*}
By the \href{http://en.wikipedia.org/wiki/Chinese_remainder_theorem}{Chinese
reminder theorem} the above system has only one solution for the variable $(a, b)$
over the $(p)(p - 1)$ possible pairs, therefore $(a x_{1} + b) \bmod p = d$ and
$(a x_{2} + b) \bmod p = e$
for $\approx \frac{p}{m}$ cases each.
Since $d, e$ are independent
\begin{equation*}
  ((a x_{2} + b) \bmod p = d) \wedge ((a x_{2} + b) \bmod p = e) \textrm{ for }
  \frac{p}{m}, \frac{p}{m} = \frac{p^{2}}{m^{2}}
\end{equation*}
Over all the possible choices of $(a, b)$:
\begin{equation*}
  \prob{(a x_{2} + b) \bmod p = d \wedge  (a x_{2} + b) \bmod p = e} = \frac{\frac{p^{2}}{m^{2}}}{p(p - 1)}
\end{equation*}

We now prove $\prob{h(x_{1}) = c_{1}} = \frac{1}{m}$.
Of all the possible $m$ buckets, one and only one is the one we look for:
$\prob{l \mod m = c_{1}} = \frac{1}{m}$
As we've seen by~\ref{p_over_m} at most $\frac{p}{m}$ such $l$ exist for $(a x + b) \mod p$,
therefore $\prob{h(x_i) = c_{i}} = \frac{\frac{p}{m}}{m} \approx \frac{1}{m}$
Which computes $\approx \frac{1}{m^{2}}$ and proves the assumption under the said approximation.

\subsection{Second proof}

\begin{proof}
We will show that both sides of  \eqref{equation:pairwise-independence} are approximately equal to $\frac{1}{m^2}$.

Given a hash function $h_{a,b} \in \mathcal{H}$, and two distinct inputs $x_1, x_2 \in \mathbb{N}$, let:
\begin{align*}
  r = (a x_1 + b) \bmod p\\
  s = (a x_2 + b) \bmod p
\end{align*}
Notice that $r \ne s$ because $r - s \equiv a(x_1-x_2) \pmod p $,  both $a$ and $x_1-x_2$ are nonzero modulo a prime number, and so their product is nonzero.

Since there are a total of $p(p-1)$ pairs of $(r, s)$ such that $r \ne s$ ($p^2$ choices subtracted the number of pairs where $r=s$), there is a one-to-one correspondence between pairs $(a, b)\in \mathbb{Z}_p^*\times\mathbb{Z}_p$ and pairs $(r, s)$. Therefore, for the left-hand side of \eqref{equation:pairwise-independence}:
$$\probs{h_{a,b}\in\mathcal{H}}{(h(x_1)= c_1, h(x_2) = c_2)} = \probs{r\ne s\in\mathbb{Z}_p^2}{r \bmod m = c_1, s \bmod m = c_2}$$

There are about $p / m$ values of $r$ that satisfy $r \bmod m=c_1$, and the same goes for $c_2$ and $s$. Hence, there are about $(p / m)^2$ choices for the pair $(r, s) \in \mathbb{Z}_p^2$ that satisfy $r \bmod m = c_1 \wedge s \bmod m = c_2$. Dividing the favorable cases with the all possible cases:
$$\frac{\lfloor p / m \rfloor^2}{p(p-1)}\le\prob{r \bmod m = c_1, s \bmod m = c_2}\le \frac{(\lfloor p / m \rfloor + 1)^2}{p(p-1)}$$
Thus, $\prob{h(x_1)= c_1, h(x_2) = c_2}\approx \frac{1}{m^2}$.

\vspace{10pt}
The right-hand side of \eqref{equation:pairwise-independence} is equal to  $\frac{1}{m^2}$, because both probabilities are $\frac{1}{m}$. Take for example $\prob{h(x_2) = c_2}$: there are about $p / m$ values of $s$ that satisfy $s \bmod m = c_2$, out of $p$ values for $s$. Hence:
$$\prob{h(x_2) = c_2} = \prob{s \bmod m = c_2} \approx \frac{\frac{p}{m}}{p} = \frac{1}{m}$$


\end{proof}