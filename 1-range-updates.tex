\section{Range updates}

Consider an array \emph{C} of n integers, initially all equal to zero.
We want to support the following operations:
\begin{itemize}
    \item \textbf{update(i, j, c):} where $0 \leq i \leq j \leq n - 1$ and $c$ is
    an integer: it changes $C$ such that $C[k] = C[k] + c$ for every $i \leq k \leq j$.
    \item \textbf{query(i)} where $0 \leq i \leq n - 1$: it returns the value of $C[i]$.
    \item \textbf{sum(i,j)} where $0 \leq i \leq j \leq n - 1$: it returns
                    $\Sigma_{k = 1}^{j}(C[k])$.
\end{itemize}

Design a data structure that uses $O(n)$ space and implements each operation above
in $O(\log(n))$ time. Note that $query(i) = sum(i, i)$ but it helps to reason.
[Hint to further save space: use an implicit tree such as the Fenwick tree (see wikipedia).]

\subsection{Solution 1: Segment tree lazy a-b sums}
Let $T$ be a segmented binary tree over a continuous interval $I: [0, N - 1]$
s.t.\ its leafs are the points in I, and the parent of two nodes comprises of their interval:
$$  n' \cup n'' = n, n' \cap n'' = \emptyset   \textrm{ s.t. } n \text{ is the parent of } n', n''$$

$T$ will keep track of the prefix sums for every interval.
We define a function
\begin{equation}
    s': [0, n - 1] \to \mathbb{N}
\end{equation}
that given a node in $T$ returns the value associated with $I$, namely the
cumulative sum of that interval.

In order to reduce the computational cost, we introduce a lazy algorithm
that doesn't propagate sums over $T$ as they are streamed in the input,
which means $s'(i)$ might not be accurate at a given time $t$ for any of the
requested operation.

We'll instead either compute over $T$ or update $T$ as necessary.
Let us define a function to do so:
\begin{equation}
    l: \mathbb{N} \to (\mathbb{N} \cup \{\epsilon\}, \mathbb{N})
\end{equation}
to keep track of our lazy sums:
\begin{equation}
    s(n) = \begin{cases}
            \epsilon, \_            &   \textrm{if no lazy prefix sum is in that interval} \\
            k, m                    &   \textrm{if a lazy sum of k is to be propagated to m}\\
            \end{cases}
\end{equation}
The \textsc{query} function is then trivial:

\begin{algorithmic}[1]
    \Function{query}{$I$, $i$, $sum$}:
    \If{$I.size = 1$}                       \Comment{Return found value}
      \State \Return $I.sum$
    \EndIf
    \If{lazy(I), $i \in I.left, i \notin I.right$}     \Comment{Lazy on left child}
      \State $lazy(I) \gets False$
      \State \Call{query}{$I.left$, $i$, $sum + I.sum$}
    \EndIf
    \If{lazy(I), $i \in I.right, i \notin I.left$} \Comment{Lazy on right child}
      \State $lazy(I) \gets False$
      \State \Call{query}{$I.right$, $i$, $sum + I.sum$}
    \EndIf
    \If{lazy(I), $i \in I.right, i \in I.left$}      \Comment{Lazy on both}
      \State $lazy(I) \gets False$
      \State \Call{query}{$I.right$, $i$, $j$, $sum + I.sum$} +
            \Call{query}{$I.left$, $i$, $j$, $sum + I.sum$}
    \EndIf
    \If{!lazy(I), $i \in I.left$}             \Comment{Not lazy on left child}
      \State \Call{query}{$I.left$, $i$, $sum$}
    \EndIf
    \If{!lazy(I), $i \in I.right$}        \Comment{Not lazy on right child}
      \State \Call{query}{$I.right$, $i$, $sum$}
    \EndIf
    \If{!lazy(I), $i \in I.right, i \in I.left$}         \Comment{Not lazy both}
      \State \Call{sum}{$I.right$, $i$, $sum$}
    \EndIf
    \EndFunction
\end{algorithmic}

\begin{algorithmic}[1]
    \Function{sum}{$I$, $i$, $j$, $sum$}:
    \If{$I.size = 1$}                                   \Comment{Return}
      \State \Return{$I.sum + sum$}\;
    \EndIf
    \If{lazy(I), $i \in I.left, i \notin I.right$}     \Comment{Lazy on left}
      \State $lazy(I) \gets False$
      \State \Call{sum}{$I.left$, $i$, $j$, $sum + I.sum$}
    \EndIf
    \If{lazy(I), $i \in I.right, i \notin I.left$} \Comment{Lazy on right}
      \State $lazy(I) \gets False$
      \State \Call{sum}{$I.right$, $i$, $j$, $sum + I.sum$}
    \EndIf
    \If{lazy(I), $i \in I.right, i \in I.left$}      \Comment{Lazy on both}
      \State $lazy(I) \gets False$
      \State \Call{sum}{$I.right$, $i$, $j$, $sum + I.sum$} +
                \Call{sum}{$I.left$, $i$, $j$, $sum + I.sum$}
    \EndIf
    \If{!lazy(I), $i \in I.left$}                        \Comment{Not lazy on both}
      \State \Call{sum}{$I.left$, $i$, $sum$}
    \EndIf
    \If{!lazy(I), $i \in I.right$}                       \Comment{Not lazy on both}
      \State \Call{sum}{$I.right$, $i$, $sum$}
    \EndIf
    \If{!lazy(I), $i \in I.right, i \in I.left$}         \Comment{Not lazy on both}
      \State \Call{sum}{$I.right$, $i$, $sum$}
    \EndIf
    \EndFunction
\end{algorithmic}

\begin{algorithmic}[1]
    \Function{update}{$I$, $i$, $j$, $k$}:
    \If{$I.size = 1$}                                   \Comment{Return}
        \State \Return $I.val \gets I.val + update$\;
        \State \Return $I.val += update$\;
    \EndIf
    \If{lazy(I), $i \in I.left, i \notin I.right$}     \Comment{Lazy on left}
        \State $lazy(I.left) \gets True$
        \State $I.left.val \gets k$
    \EndIf
    \If{lazy(I), $i \in I.right, i \notin I.left$} \Comment{Lazy on right}
        \State $lazy(I.right) \gets True$
        \State $I.right.val \gets k$
    \EndIf
    \If{lazy(I), $i \in I.right, i \in I.left$}      \Comment{Lazy on both}
        \State $lazy(I) \gets True$
        \State $I.val \gets k$
    \EndIf
    \If{!lazy(I), $i \in I.left$}                        \Comment{Not lazy}
        \State \Call{update}{$I.left$, $i$, $update$}
        \State update($I.left$, $i$, $update$)
    \EndIf
    \If{!lazy(I), $i \in I.right$}                       \Comment{Not lazy}
        \State \Call{update}{$I.right$, $i$, $update$}
        \State update($I.right$, $i$, $update$)
    \EndIf
    \If{!lazy(I), $i \in I.right, i \in I.left$}         \Comment{Not lazy}
        \State \Call{update}{$I.right$, $i$, $update$}
        \State update($I.right$, $i$, $update$)
    \EndIf
    \EndFunction
\end{algorithmic}
