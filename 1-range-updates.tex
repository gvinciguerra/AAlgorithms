\section{Range updates}

Consider an array $C$ of $n$ integers, initially all equal to zero. We want to support the following operations:
\begin{itemize}
  \item $update(i, j, c)$, where $0 \leq i \leq j \leq n - 1$ and $c$ is an integer: it changes $C$ such that $C[k] := C[k] + c$ for every $i \leq k \leq j$.
  \item $query(i)$, where $0 \leq i \leq n - 1$: it returns the value of $C[i]$.
  \item $sum(i,j)$, where $0 \leq i \leq j \leq n - 1$: it returns $\sum_{k = 1}^j C[k]$.
\end{itemize}

Design a data structure that uses $O(n)$ space, takes $O(n \log n)$ construction time, and implements each operation above in $O(\log n)$ time. Note that $query(i) = sum(i, i)$ but it helps to reason. [Hint: For the general case, use the segment tree seen in class, which uses $O(n \log n)$ space: prove that its space is actually $O(n)$ when it is employed for this problem.]

\subsection{First solution}

Let $T$ be a segmented binary tree over a continuous interval $I: [0, N -1]$ s.t. $2^{height(T) - 1} = N$, that is, $T$ stores $N$ singleton interval in its leaves.
Then $T \in O(n)$:
\begin{equation*}
\abs{T - \text{last\_level}} = \sum_{i=0}^{height(T) - 2}2^i = 2^{height(T) - 1} - 1 = N - 1
\end{equation*}
therefore
\begin{equation*}
\abs{T} = N + N - 1 = 2N - 1 = 2N
\end{equation*}

We build $2$ isomorphic trees $S$ and $L$ s.t. $\abs{T} + \abs{S} + \abs{L} = 6n \in O(n)$ defined as \emph{sum tree} and \emph{lazy tree} respectively.
$S.node$ will hold the cumulative sum for the interval it represents, while $L.node$ will hold a \emph{lazy} value of the sum, not propagated to the sub-intervals of the given node.
That is, $sum(node) = S.node + L.node$.
With this in mind, let us define \emph{sum} and \emph{update}.
\begin{itemize}
\item \textbf{update(i, j, k):} We operate on an half-lazy basis: as updating each node comprising the query interval, we split it "to the right": given an interval $[i, j] \exists l \in (i, j) \text{ s.t. } T.node.interval\_end = l, node \in T$.
That is, we can always split a given interval $I'$ in an interval sharing its starting position and comprising part of the interval.
As we have singleton intervals in the leaves, this is trivial to show.
We are able to apply the above updating the values of $S$ and $L$ as we traverse the tree: starting from the root, update $S.node = k * I.size$ (node = root at the first iteration); pick the $S.Rchild$ child "to the right" of the interval and update $L.Rchild = k$, and apply the recursive call on the $S.Lchild$ child "to the left".
We can easily reverse this procedure with a "to the left" policy.
Now, as the $S.Rchild$ update comprised the whole interval, each node receives its update in the top-down traversal, while $S.Lchild$ is taken care of recursively until it reaches the interval maximum, which is updated in $S$ and its right sibling which is updated in $L$.
As at each level we split in half, we have $\log N$ operations on update.
\item \textbf{sum(i, j):} We sum as stated above by traversing the tree and cumulating lazy values in the traversal, summing $S.node$ as we find nodes in the interval.
Given that this is a trivial traversal of a binary tree, $\log N$ operations are involved (actually $\leq 2N$ as also lazy values are summed).
Query is a trivial \textbf{sum(i, i)}, the same holds.
\end{itemize}

%Let $T$ be a segmented binary tree over a continuous interval $I: [0, N - 1]$
%s.t.\ its leafs are the points in I, and the parent of two nodes comprises of their interval:
%$$  n' \cup n'' = n, n' \cap n'' = \emptyset   \textrm{ s.t. } n \text{ is the parent of } n', n''$$
%
%$T$ will keep track of the prefix sums for every interval.
%We define a function
%\begin{equation}
%    s': [0, n - 1] \to \mathbb{N}
%\end{equation}
%that given a node in $T$ returns the value associated with $I$, namely the
%cumulative sum of that interval.
%
%In order to reduce the computational cost, we introduce a lazy algorithm
%that doesn't propagate sums over $T$ as they are streamed in the input,
%which means $s'(i)$ might not be accurate at a given time $t$ for any of the
%requested operation.
%
%We'll instead either compute over $T$ or update $T$ as necessary.
%Let us define a function to do so:
%\begin{equation}
%    l: \mathbb{N} \to (\mathbb{N} \cup \{\epsilon\}, \mathbb{N})
%\end{equation}
%to keep track of our lazy sums:
%\begin{equation}
%    s(n) = \begin{cases}
%            \epsilon, \_            &   \textrm{if no lazy prefix sum is in that interval} \\
%            k, m                    &   \textrm{if a lazy sum of k is to be propagated to m}\\
%            \end{cases}
%\end{equation}
%The \textsc{query} function is then trivial:
%
%\begin{algorithmic}[1]
%    \Function{query}{$I$, $i$, $sum$}:
%    \If{$I.size = 1$}                       \Comment{Return found value}
%      \State \Return $I.sum$
%    \EndIf
%    \If{lazy(I), $i \in I.left, i \notin I.right$}     \Comment{Lazy on left child}
%      \State $lazy(I) \gets False$
%      \State \Call{query}{$I.left$, $i$, $sum + I.sum$}
%    \EndIf
%    \If{lazy(I), $i \in I.right, i \notin I.left$} \Comment{Lazy on right child}
%      \State $lazy(I) \gets False$
%      \State \Call{query}{$I.right$, $i$, $sum + I.sum$}
%    \EndIf
%    \If{lazy(I), $i \in I.right, i \in I.left$}      \Comment{Lazy on both}
%      \State $lazy(I) \gets False$
%      \State \Call{query}{$I.right$, $i$, $j$, $sum + I.sum$} +
%            \Call{query}{$I.left$, $i$, $j$, $sum + I.sum$}
%    \EndIf
%    \If{!lazy(I), $i \in I.left$}             \Comment{Not lazy on left child}
%      \State \Call{query}{$I.left$, $i$, $sum$}
%    \EndIf
%    \If{!lazy(I), $i \in I.right$}        \Comment{Not lazy on right child}
%      \State \Call{query}{$I.right$, $i$, $sum$}
%    \EndIf
%    \If{!lazy(I), $i \in I.right, i \in I.left$}         \Comment{Not lazy both}
%      \State \Call{sum}{$I.right$, $i$, $sum$}
%    \EndIf
%    \EndFunction
%\end{algorithmic}
%
%\begin{algorithmic}[1]
%    \Function{sum}{$I$, $i$, $j$, $sum$}:
%    \If{$I.size = 1$}                                   \Comment{Return}
%      \State \Return{$I.sum + sum$}\;
%    \EndIf
%    \If{lazy(I), $i \in I.left, i \notin I.right$}     \Comment{Lazy on left}
%      \State $lazy(I) \gets False$
%      \State \Call{sum}{$I.left$, $i$, $j$, $sum + I.sum$}
%    \EndIf
%    \If{lazy(I), $i \in I.right, i \notin I.left$} \Comment{Lazy on right}
%      \State $lazy(I) \gets False$
%      \State \Call{sum}{$I.right$, $i$, $j$, $sum + I.sum$}
%    \EndIf
%    \If{lazy(I), $i \in I.right, i \in I.left$}      \Comment{Lazy on both}
%      \State $lazy(I) \gets False$
%      \State \Call{sum}{$I.right$, $i$, $j$, $sum + I.sum$} +
%                \Call{sum}{$I.left$, $i$, $j$, $sum + I.sum$}
%    \EndIf
%    \If{!lazy(I), $i \in I.left$}                        \Comment{Not lazy on both}
%      \State \Call{sum}{$I.left$, $i$, $sum$}
%    \EndIf
%    \If{!lazy(I), $i \in I.right$}                       \Comment{Not lazy on both}
%      \State \Call{sum}{$I.right$, $i$, $sum$}
%    \EndIf
%    \If{!lazy(I), $i \in I.right, i \in I.left$}         \Comment{Not lazy on both}
%      \State \Call{sum}{$I.right$, $i$, $sum$}
%    \EndIf
%    \EndFunction
%\end{algorithmic}
%
%\begin{algorithmic}[1]
%    \Function{update}{$I$, $i$, $j$, $k$}:
%    \If{$I.size = 1$}                                   \Comment{Return}
%        \State \Return $I.val \gets I.val + update$\;
%        \State \Return $I.val += update$\;
%    \EndIf
%    \If{lazy(I), $i \in I.left, i \notin I.right$}     \Comment{Lazy on left}
%        \State $lazy(I.left) \gets True$
%        \State $I.left.val \gets k$
%    \EndIf
%    \If{lazy(I), $i \in I.right, i \notin I.left$} \Comment{Lazy on right}
%        \State $lazy(I.right) \gets True$
%        \State $I.right.val \gets k$
%    \EndIf
%    \If{lazy(I), $i \in I.right, i \in I.left$}      \Comment{Lazy on both}
%        \State $lazy(I) \gets True$
%        \State $I.val \gets k$
%    \EndIf
%    \If{!lazy(I), $i \in I.left$}                        \Comment{Not lazy}
%        \State \Call{update}{$I.left$, $i$, $update$}
%        \State update($I.left$, $i$, $update$)
%    \EndIf
%    \If{!lazy(I), $i \in I.right$}                       \Comment{Not lazy}
%        \State \Call{update}{$I.right$, $i$, $update$}
%        \State update($I.right$, $i$, $update$)
%    \EndIf
%    \If{!lazy(I), $i \in I.right, i \in I.left$}         \Comment{Not lazy}
%        \State \Call{update}{$I.right$, $i$, $update$}
%        \State update($I.right$, $i$, $update$)
%    \EndIf
%    \EndFunction
%\end{algorithmic}
%
\subsection{Second solution}

We use a segment binary tree $T_I$ over the interval $I = [0, n - 1]$, that is, a tree whose leaves are the points in $I$ and the parent of two nodes is the union of their interval. More formally, if $x.interval$ denotes the attribute $interval$ of the node $x$,
\begin{enumerate}
  \item $l.interval \cup r.interval = n.interval \iff n \text{ is the parent of $l$ and $r$}$;
  \item $\text{if $x$ and $y$ are leaves, then } x.interval \cap y.interval = \emptyset \wedge |x.interval|=|y.interval|=1$.
\end{enumerate}
For example, the segment tree for $I=[0,7]$ is the following:
\begin{center}
  \begin{tikzpicture}[sibling distance=10pt]
    \tikzstyle{every node}=[circle,draw]
    \Tree [.0-7 [.0-3 [.0-1 [.0 ]  [.1 ] ] [.2-3 [.2 ] [.3 ] ] ] [.4-7 [.4-5 [.4 ] [.5 ] ] [.6-7 [.6 ] [.7 ] ] ] ]
  \end{tikzpicture}
\end{center}

We associate with each node $x$ of $T_I$ some attributes: $x.sum$, that stores $\sum_{i \in x.interval} C[i]$; and $x.lazy$, that stores a value that need to be propagated to each descendant of $x$. This means that $x.sum$ might not be accurate at a given time for any of the requested operation.  

\paragraph{Range operations.} In both operations we traverse the tree recursively starting from the root and, at \emph{each} recursive step on any internal node $x$, if $x.lazy \neq 0$: we set $x.sum \gets x.sum + |x.interval| \times x.lazy$, we propagate the lazy information to $x$'s children $x.left.lazy \gets x.lazy$, $x.right.lazy \gets x.lazy$ and, finally, we reset the information $x.lazy \gets 0$. Afterwards, if the operation is 
\begin{itemize}
  \item $sum(i,j)$, we do the following:
  \begin{enumerate}
    \item if $x.interval \cap [i,j] = \emptyset$, we return 0;
    \item if $x.interval \subseteq [i,j]$, we return $x.sum$;
    \item otherwise we repeat the procedure $sum(i,j)$ on $x.left$ and $x.right$, returning the sum of these calls.
  \end{enumerate}

  \item $update(i,j,c)$, we do the following:
  \begin{enumerate}
    \item if $x.interval \cap [i,j] = \emptyset$, we stop the recursion on this subtree;
    \item if $x.interval \subseteq [i,j]$, we update $x.sum \gets |x.interval| \times c$, and $x.left.lazy \gets x.left.lazy + c$, $x.right.lazy \gets x.right.lazy + c$;
    \item otherwise we repeat the procedure $update(i,j,c)$ on $x.left$ and $x.right$.
  \end{enumerate} 
\end{itemize}
The space occupied by $T_I$ is $\sum_{i=0}^{\log_2 n} n/2^{-i}=2n-1$.