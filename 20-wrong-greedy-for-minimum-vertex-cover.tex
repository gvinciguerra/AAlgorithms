\section{Wrong greedy for minimum vertex cover}

Find an example of (family of) graphs for which the following greedy approach fails to give a 2-approximation for the minimum vertex cover problem (and prove why this is so). 
Start out with an empty $\tilde{S}$.
Choose each time a vertex $v$ with the largest number of incident edges in the current graph.
Add $v$ to $\tilde{S}$ and remove its incident edges.
Repeat the process on the resulting graph as long as there are edges in it.
Return $\abs{\tilde{S}}$ as the a approximation of the minimal size of a vertex cover for the original input graph.
Generalize your argument to show that the above greedy algorithm cannot actually provide an $r$-approximation for any given constant $r >1$.

\paragraph{Solution}
We create a class of graphs $G = (V, E), \abs{V} = N$ s.t. $V$ is partitioned in $S, R$ s.t. $\abs{S} = k, \abs{R} = \sum_{i = 1}^{k} \left(\lfloor{\frac{k}{i}} \rfloor \right)$ respectively and define them as \emph{senders} and \emph{receivers}.

We then build the edges in the following way in order to obtain a minimum cover $L$:
	\begin{enumerate}
	\item For the current iteration $i$, if $\lfloor{\frac{k}{i}}\rfloor = 0$ we stop.
	\item Split the current slice $R$, consider the first $\lfloor{\frac{k}{i}}\rfloor$ vertexes: let $R^i$ be this slice.
	\item For every vertex $v \in R^i$ build $\frac{m}{\lfloor{\frac{k}{i}}\rfloor}$ outgoing edges to each of the $m$ nodes $\in R$.
	We can then repeat step $1$ with $R \gets R \setminus R^i$.
	\end{enumerate}
Once the above procedure ends, we have partitions vertexes of size $k \in S: deg(v) = 1$, one partition of vertexes of size $\lfloor{\frac{k}{2}}\rfloor \in S: deg(v) = \lfloor{\frac{k}{2}}\rfloor \forall v \in S$, etc. 
Each node in $R$ will have an incoming edge for every node $n \in L$ and will therefore be in the minimum vertex cover, as by hypothesis $\abs{S} > \abs{R}$
The greedy algorithm then starts by removing the highest-degree node: we find this on top of $R$, as it was doubled at each iteration.
By construction we have $k$ \emph{senders} partitions whose highest degree is $k$ (note as the degree increases in the summation up to $\lfloor{\frac{k}{i}}\rfloor$ for $i = k$).
The \emph{receivers} have a maximum degree of $\sum_{i = 1}^{k} \lfloor{\frac{k}{i}}\rfloor \approx k \ln k$,
as each of them received an incoming edge from each partitions $\in S$ that we asserted being $\lfloor{\frac{k}{i}}\rfloor$ by construction.
The greedy algorithm will start removing from the vertex with highest degree: it will be forced to remove from the last built slices up to $\lfloor{\frac{k}{i}}\rfloor > \frac{m}{\lfloor{\frac{k}{i}}\rfloor}$, thus cutting at least $k \ln k > m$ vertexes before cutting in the minimum cover $S$.
We can build a family of such graphs by making $k$ grow at will: as $k \ln k > r \forall k$ such family will have an increasingly larger \emph{r-approximation}:
\begin{equation*}
  k \ln k > rk \text{ }\forall k, r
\end{equation*}