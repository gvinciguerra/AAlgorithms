\section{Hashing sets}

Your company has a database $S \subseteq U$ of keys. For this database, it uses
a randomly chosen hash function $h$ from a universal family $H$ (as seen in class);
it also keeps a bit vector $B_S$ of $m$ entries, initialized to zeroes, which are
then set $B_S[h(k)] = 1 \forall k \in S$ (note that collisions may happen).
Unfortunately, the database has been lost, thus only BS and h are known, and the
rest is no more accessible.
Now, given $k \in U$, how can you establish if $k$ was in $S$ or not?
What is the probability of error? (Optional: can you estimate the size $|S|$ of
$S$ looking at h and $B_S$ and what is the probability of error?)
Later, another database R has been found to be lost: it was using the same hash
function h, and the bit vector BR defined analogously as above.
Using $h, B_S, B_R$, how can you establish if $k$ was in $S \cap R$ (union), $S \cup R$
(intersection), or $S \\ R$ (difference)? What is the probability of error?

\subsection{$k \in S$}
Trivially for $B_s[h(k)] = 0$ we can answer \textsc{false} with $\prob(error) = 0$.
Let us analyse the opposite case, $B_s[h(k)] = 1$.
Let $i \in [0,m]$ be some index s.t. $B_s[i] = 1$, and let $cl_{S}(i)$ be the list of
$k \in S: h(k) = 1$ for some set $S \in \displaystyle {\mathcal {P}}(S)$.
We can then denote the sets $cl_{U} := {k_{U} : k_{U} \in U, h(k) = i},
cl_{S} := {k_{S} : k_{S} \in S, h(k) = i}$; it is trivial to show that
    \begin{itemize}
    \label{6_cl_inclusion} \item $cl_{S} \subseteq cl_{U}$ as $S \in U$.
    \label{6_cl_length} \item $| cl_{S}(k) | \leq | cl_{U}(k) | \forall k \in U$ as $S \in U$.
    \end{itemize}
Let us not try and estimate $|cl_{S}(k)|$: given $h$ is universal, we have an
expected value of collisions of $\expect[X_{k}] \approx \frac{1}{m} \forall k \in S$,
that is
\begin{align*}
    \prob(h(k^{0}) = c) = \frac{1}{m}                               \\
    \prob(h(k^{1}) = c) = \frac{1}{m^{2}}                           \\
    \prob(h(k^{i - 1}) = c \and h(k^{i} = c) = \frac{1}{m^{i}}      \\
    \prob(h(k) = c, \forall k \in S) = \frac{1}{m^{|S|}}
\end{align*}
We can similarly compute the probability of not collision by simply replacing
$\frac{1}{m}$ with $(1 - \frac{1}{m}): \prob(h(k) != c, \forall k \in S) = 1 - \frac{1}{m^{|S|}}$.

Given our estimate of the collision list, we can now compute an estimate of the
error probability: by~\ref{6_cl_inclusion}, we give an erroneous answer whenever
$k \in cl_{U}(k), k \notin cl_{S}(k)$, that is we have a margin of error of
$cl_{U}(k) \setminus cl_{S}(k)$ whose size is $|cl_{U}(k)| - |cl_{S}(k)|$.
Given the set of $k$ for which $h(k) = i$ the \emph{bad answers} are then
    \begin{equation}
    1 - \prob(\textrm{good answer}) = 1 - {(1 - \frac{1}{m})}^{|S|}
    \end{equation}
We now provide a lower and upper bound for the said value.
The lower bound is given by the \emph{perfect hash} with no collisions: given
$e = $ number of $1$ in $B_s$, $e$ is the lowerbound.
Provided that  $m \geq c |S|$ for some $c > 1, |S| \leq \frac{m}{c}$:
    \begin{equation}
    \prob(\textrm{collision over i}) = \alpha_{S} = \frac{\frac{m}{c}}{m} = \frac{1}{c}
    \end{equation}
Therefore
    \begin{equation}
    e \leq 1 - \frac{1}{m^{|S|}} \leq \frac{1}{c}
    \end{equation}

\subsection{Expected number of 1 in $B_{s}$}
We define a random variable $X_{k}$:
    \begin{equation*}
    X_{k} = \begin{cases}
            1   & if B_{s}[k] = 1  \\
            0   & otherwise
            \end{cases}
    \end{equation*}
and one over the ones in $B_{s}$, $X$:
    \begin{equation}
        X = \Sigma_{k = 0}^{m - 1}X_k
    \end{equation}.
We compute the relative expected value $\expect{X}$:
    \begin{align*}
       \expect{X} & = \expect{\Sigma_{k = 0}^{m - 1}X_k} \\
       & = \Sigma_{k = 0}^{m - 1}\prob(B_{s}[k] = 1) \\
       & = \Sigma_{k = 0}^{m - 1}\frac{\alpha}{m} \leq \frac{m}{c}
   \end{align*}
where $\prob(B_{s}[k] = 1) = \frac{\alpha}{m}$ as $\alpha$ are the favorable cases
(i.e. the collisions for a generic bucket $B_s[i]$), and $m$ is the size of $B_s$.
