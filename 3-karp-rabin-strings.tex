\section{Karp-Rabin fingerprinting on strings}

Given a string $S \equiv S[0 \dots n - 1]$, and two positions $0 \leq i < j \leq n - 1$, the longest common extension $lce_S(i, j)$ is the length of the maximal run of matching characters from those positions, namely: if $S[i] \neq S[j]$ then $lce_S(i, j) = 0$; otherwise, $lce_S(i, j) = \max\{l \geq 1 : S[i \dots i + l - 1] = S[j \dots j + l - 1]\}$. For example, if $S = \texttt{abracadabra}$, then $lce_S(1, 2) = 0$, $lce_S(0, 3) = 1$, and $lce_S(0, 7) = 4$. Given $S$ in advance for preprocessing, build a data structure for $S$ based on the Karp-Rabin fingerprinting, in $O(n \log n)$ time, so that it supports subsequent online queries of the following two types:
\begin{itemize}
  \item $lce_S(i,j)$: it computes the longest common extension at positions $i$ and $j$ in $O(\log n)$ time.
  \item $equal_S(i,j,l)$: it checks if $S[i \dots i+l-1] = S[j \dots j+l-1]$ in constant time.
\end{itemize}
Analyze the cost and the error probability. The space occupied by the data structure can be $O(n \log n)$ but it is possible to use $O(n)$ space. [Note: in this exercise, a one-time preprocessing is performed, and then many online queries are to be answered on the fly]

\subsection{First solution}

In order to save computational cycles on checks over ranges we use a similar structure
to the one in the range updates: we compute the hashing on the first character in $O(1)$
time, then roll the hash through the $n - 1$ remaining characters through $n O(1)$
operations.
We call $H$ this array; we also denote $h_k$ as the function $c a^{i}$ computating
the Rabin-Karph hash of a string $s$.
The reader shall now see that $\exists h^{-1}(s)$: that is, $h$ is invertible in $O(1)$.
The entries $h[i] = \sum_{i \in [0, n - 1]}(h(i))$ have cumulative hash and the following
properties hold:
    \begin{itemize}
    \item $h[s[i]] = (h[i] - h[i - 1]) / a^{-1}, a^{-1} = a^{1}$
    \item $h[i..j] - h[k..l] = (h[l] - h[k - 1]) / a^{-1} -
            (h[j] - h[i - 1]) / a^{-1}, a^{-1} = \textrm{modular inverse}$
    \end{itemize}

\textsc{equals} works on cumulative hashes, subtracting them and scaling them
accordingly, as our $rabin$ function multiplies by an $a^{i}$ costant.

\begin{algorithmic}[1]
  \Function{equals}{$i$, $j$, $length$}:
    \State $h_i = h[i + length] - h[i - 1]$\;
    \State $h_j = h[j + length] - h[j - 1]$\;

    \State $h^{i} = h_i / inv(a, i, l)$\;
    \State $h^{j} = h_j / inv(a, j, l)$\;

    \Return{$h^{i} - h^{j} == 0$}\;
    \EndFunction

    \Function{inv}{$h$, $k$, $l$}:
    \Return $h^{k - l}$
    \EndFunction
\end{algorithmic}

\textsc{lce} works on cumulative hashes, checks the equality on the middle element
of the strings and runs recursively on the half with different hashing.
We define \textsc{lce} as an auxiliary function

\begin{algorithmic}[1]
  \Function{lce}{$i$, $j$, $l$}:
    \State eq = \Call{equals}{$i$, $j$}\;

    \If{eq}
        \Return{$l$}
    \ElsIf{$\neg$ \Call{equals}{$(j - i) / 2$, $(n - j) / 2$, $l$}}
        \Return{\Call{equals}{$(j - i) / 2$, $(n - j) / 2$}}                            % First half
        \Else $(j - i) / 2 + $ \Return{\Call{equals}{$(j - i) / 2$, $(n - j) / 2$}}     % Second half
    \EndIf

    \State $h_i = h[i + length] - h[i - 1]$\;
    \State $h_j = h[j + length] - h[j - 1]$\;

    \State $h^{i} = h_i / inv(a, i, l)$\;
    \State $h^{j} = h_j / inv(a, j, l)$\;

    \Return{$h^{i} - h^{j} == 0$}\;
    \EndFunction

    \Function{inv}{$h$, $k$, $l$}:
    \Return $h^{k - l}$
    \EndFunction
\end{algorithmic}

\subsection{Second solution}

We interpret the string $S \in \Sigma^*$ as the polynomial $S(z)=\sum_{i=0}^{n-1}S[i]z^i$, where $S[i]$ is a symbol encoded with a number in $\{0, \dots, |\Sigma|-1\}$. Given a prime number $p$ greater than both $2n$ and $|\Sigma|$, we define the fingerprint on strings $h_p$ as:
$$h_p(S)=h_p\left(\sum_{i=0}^{n-1}S[i] \cdot z^i\right) = \left(\sum_{i=0}^{n-1}S[i]  \cdot z^i\right) \bmod p$$
were $z$ is randomly chosen from $\mathbb{Z}_p$.

If two string $A$ and $B$ are equal, then also their fingerprints are equal. If they are different, the probability that their fingerprints are the same is at most $\frac{n}{p}<\frac{n}{2n}=\frac{1}{2}$, because there are $p$ choices for $z$ and at most $n$ roots of the polynomial $P(z)=A(z)-B(z)$ (follows from the  fundamental theorem of algebra and the fact that $A(z)$ and $B(z)$ have both degree $n$) \cite[169]{Motwani10}.

\paragraph{Queries.} To implement efficiently the queries on $S$, we use an array $H$ with the hashes of the string's prefixes, formally: 
$$H[i]=h_p(S[0..i]) \quad \forall i \in [0, n-1]$$
where $S[i..j]$ denotes the substring $S[i]S[i+1] \dots S[j]$.

With the array $H$ we can compute in constant time the fingerprint of any substring of $S$, hence we can decide --- with a probability of error --- $equal_S(i,j,l)$, by comparing the substrings' fingerprints. To compute $lce_S(i,j)$, we search with $O(\log n)$ different values of $l$ inside the interval $[0, \min(j-i,n-j)]$ until the result is found. Listing \ref{listing:lce} shows an implementation in Python.

\paragraph{Analysis.} $H$ is built in $O(n)$ time, because at each iteration the fingerprint in $H[i]$ is computed in constant time from the fingerprint in $H[i-1]$. The space needed by the solution is $O(n \log p)$, because we need to store the array $H$, that has $n$ entries of size $\log_2 p$. 

\begin{lstlisting}[frame=single,language=Python,caption={Source code to compute the longest common extension efficiently.},label=listing:lce]
p = 179426549
z = 3
s = "abracadabra"

def h(s):
    return sum(ord(c) * z ** i for i, c in enumerate(s)) % p

H = [ord(s[0]) % p] # Array with hashes of prefixes
exp = z
for i in range(1, len(s)): # O(n)
    H.append((H[i-1] + (ord(s[i]) * exp) % p))
    exp = exp * z

def rollLeft(j, count): # Compute the hash of the string s[j:j+count]
    assert count > 0 and j + count <= len(s)
    if j == 0:
        return H[count-1]
    return (H[j+count-1] - H[j - 1]) / z ** j

def equal(i, j, l): # O(1)
    assert i < j and j + l - 1 < len(s) and l > 0
    hs1 = rollLeft(i, l) # hash of s[i:i+l]
    hs2 = rollLeft(j, l) # hash of s[j:j+l]
    return hs1 == hs2

def lceaux(i, j, intsize, maxlce): # O(logn)
    if (intsize <= 1):
        return 1 if equal(i, j, 1) else 0
    if equal(i, j, intsize):  # Check if there is a longer common extension
        if (intsize + intsize / 2 >= maxlce):
            return maxlce
        return lceaux(i, j, intsize + intsize / 2, maxlce)
    else: # Check if there is a shorter common extension
        return lceaux(i, j, intsize - intsize / 2, intsize)

def lce(i, j):
    maxlce = min(j - i, len(s) - j)
    if (equal(i, j, maxlce)):
        return maxlce
    return lceaux(i, j, maxlce, maxlce)
\end{lstlisting}

