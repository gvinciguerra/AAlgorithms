\section{Karp-Rabin fingerprinting on strings}

Given a string $S: |S| = n$, and two positions $0 \leq i < j \leq (n - 1)$,
the longest common extension $lceS(i, j)$ is the length of the maximal run of matching
characters from those positions, namely: if $S[i] 6= S[j]$ then $lceS(i, j) = 0$;
otherwise, $lceS(i, j) = \max{l \geq 1 : S[i ... i + l - 1] = S[j ... j + i - 1]}$.
For example, if S = abracadabra, then $lceS(1, 2) = 0$, $lceS(0, 3) = 1$, and
$lceS(0, 7) = 4$.
Given S in advance for preprocessing, build a data structure for S based
on the Karp-Rabin fingerprinting, in $O(n \ln(n)$) time, so that it supports subsequent
    \begin{itemize}
    \item $lceS(i, j)$: it computes the longest common extension at positions i
    \item $equals (i, j, c)$: it checks if $S[i ... i + ` - 1] = S[j ... j + ` - 1]$ in constant time.
    \end{itemize}
Analyze the cost and the error probability.
The space occupied by the data structure can be $O(n \log(n))$ but it is possible
to use $O(n)$ space.
[Note: in this exercise, a onetime preprocessing is performed, and then many online
queries are to be answered on the fly.]

\vspace{0.5cm}
\paragraph{Solution.}
In order to save computational cycles on checks over ranges we use a similar structure
to the one in the range updates: we compute the hashing on the first character in $O(1)$
time, then roll the hash through the $n - 1$ remaining characters through $n O(1)$
operations.
We call $H$ this array; we also denote $h_k$ as the function $c a^{i}$ computating
the Rabin-Karph hash of a string $s$.
The reader shall now see that $\exists h^{-1}(s)$: that is, $h$ is invertible in $O(1)$.
The entries $h[i] = \sum_{i \in [0, n - 1]}(h(i))$ have cumulative hash and the following
properties hold:
    \begin{itemize}
    \item $h[s[i]] = (h[i] - h[i - 1]) / a^{-1}, a^{-1} = a^{1}$
    \item $h[i..j] - h[k..l] = (h[l] - h[k - 1]) / a^{-1} -
            (h[j] - h[i - 1]) / a^{-1}, a^{-1} = \textrm{modular inverse}$
    \end{itemize}

\textsc{equals} works on cumulative hashes, subtracting them and scaling them
accordingly, as our $rabin$ function multiplies by an $a^{i}$ costant.

\begin{algorithmic}[1]
  \Function{equals}{$i$, $j$, $length$}:
    \State $h_i = h[i + length] - h[i - 1]$\;
    \State $h_j = h[j + length] - h[j - 1]$\;

    \State $h^{i} = h_i / inv(a, i, l)$\;
    \State $h^{j} = h_j / inv(a, j, l)$\;

    \Return{$h^{i} - h^{j} == 0$}\;
    \EndFunction

    \Function{inv}{$h$, $k$, $l$}:
    \Return $h^{k - l}$
    \EndFunction
\end{algorithmic}

\textsc{lce} works on cumulative hashes, checks the equality on the middle element
of the strings and runs recursively on the half with different hashing.
We define \textsc{lce} as an auxiliary function

\begin{algorithmic}[1]
  \Function{lce}{$i$, $j$, $l$}:
    \State eq = \Call{equals}{$i$, $j$}\;

    \If{eq}
        \Return{$l$}
    \ElsIf{$\neg$ \Call{equals}{$(j - i) / 2$, $(n - j) / 2$, $l$}}
        \Return{\Call{equals}{$(j - i) / 2$, $(n - j) / 2$}}                            % First half
        \Else $(j - i) / 2 + $ \Return{\Call{equals}{$(j - i) / 2$, $(n - j) / 2$}}     % Second half
    \EndIf

    \State $h_i = h[i + length] - h[i - 1]$\;
    \State $h_j = h[j + length] - h[j - 1]$\;

    \State $h^{i} = h_i / inv(a, i, l)$\;
    \State $h^{j} = h_j / inv(a, j, l)$\;

    \Return{$h^{i} - h^{j} == 0$}\;
    \EndFunction

    \Function{inv}{$h$, $k$, $l$}:
    \Return $h^{k - l}$
    \EndFunction
\end{algorithmic}
