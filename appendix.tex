\appendix
\section{Hogwarts}

The Hogwarts School\footnote{\url{http://didawiki.cli.di.unipi.it/lib/exe/fetch.php/magistraleinformatica/alg2/algo2_16/hogwarts.pdf}}
is modeled as a graph $G=(V, E)$ where $V$ is the set of castle's rooms and $E \subseteq V \times V$
is the set of the stairs.
Each stair is labelled with the time of appearance and disappearance, and can be
walked in both directions, therefore the graph is undirected.
The goal is to find, if possible, the minimum amount of time required to go from
the first to the last room.

\subsection{Solution 1: Preprocessing-then-Dijkstra}

Dijkstra is able to find the shortest path in a graph with non-negative weights
on its edges.
Our main idea is to create a Dijkstra compatible graph through a \textsc{normalize}
function, then apply Dijkstra to it in order to find the shortest path.
The core of the preprocessing is the \textsc{normalize} function which computes traversal
times between nodes at a given time \emph{time}:

\begin{algorithmic}[1]
  \Function{normalize}{$from$, $to$, $time$}:
    \State $t \gets \infty$
    \If{$start[v'] \leq t < end[v']$}  \Comment{No waiting time}
      \State $t \gets t + 1$
    \ElsIf{$t < start[v']$}            \Comment{Waiting time}
      \State $t \gets start[v'] + 1$
    \Else
      \State $t \gets \infty$          \Comment{Available time already expired}
    \EndIf
      \State \Return{$t$}
    \EndFunction
\end{algorithmic}

The normalize function is then applied to a node traversal:

\subsubsection{Pseudo-code}

\begin{algorithmic}[1]
  \State create vertex set $Q$ of unvisited nodes\;
  \State create vertexes set $E'$ of edges weight\;
  \State $time \gets 0$                   \Comment{Initial time for traversal}
  \State $edges \gets$ \Call{stairs\_of}{0}\;      \Comment{Get incoming/outgoing edges
   of the source node}
  \Function{process}{$node$, $time$}
    \If{$edge \in visited\_edges$}
      \State \Return{}
    \EndIf

    \State $traversal\_time \gets \infty$
    \ForAll{$neighbor \in neighbors\_of\_node$}
      \State $traversal\_time \gets$ \Call{traversal\_time}{$node$, $neighbor$, $time$}
      \State $E'[0][node] \gets traversal\_time$   \Comment{$E'[i][j]$ holds the
                                                        weight/traversal}
      \State \Comment{time for the stair between $i$ and $j$}

      \ForAll{$new\_neighbor \in neighbors\_of\_neighbor$}
        \State \Call{normalize}{$neighbor$, $new\_neighbor$, $traversal\_time$}
      \EndFor
    \EndFor
    \If{\Call{dijkstra}{$V, E'$} = $\infty$}
    	\State \Return{-1}
    \Else
      \State \Return{$t$}
    \EndIf
  \EndFunction
\end{algorithmic}

\begin{framed}
  \noindent
  \textbf{Computational cost}: $\Theta(n^{2})$ if the vertex set in \textsc{dijkstra} is implemented
  as an array. $O(|E|+|V|\log |V|)$ with Fibonacci heap.
\end{framed}

\subsection{Solution 2: HogwartsDijkstra}

\begin{algorithmic}[1]
  \Function{HogwartsDijkstra}{$G$}:
  \State create vertex set $Q$ of unvisited nodes
  \ForAll{vertex $v \in V$}      \Comment{initialization}
      \State $time[v] \gets \infty$  \Comment{unknown time from source to v}
      \State add $v$ to $Q$          \Comment{all nodes initially in Q}
  \EndFor
  \State $time[0] \gets 0$ \Comment{time from source to source}
  \While{$Q\ne \emptyset$}
      \State $u \gets x \in Q$ with $\min \{time[x]\}$
      \State remove $u$ from $Q$
      \ForAll{neighbor $v$ of $u$}:
          \If{$time[u] \leq appear[v]$}
              \State $alt \gets appear[v] + 1$ \Comment{wait the appearance of
                                                                    the stair}
          \ElsIf{$time[u] < disappear[v]$}
              \State $alt \gets time[u] + 1$       \Comment{use the stair}
          \Else
              \State $alt \gets \infty$            \Comment{the stair has
                                                        already disappeared}
          \EndIf
          \If{$alt < time[v]$}
              \State $time[v] \gets alt$           \Comment{a quicker path to
                                                            $v$ has been found}
          \EndIf
      \EndFor
  \EndWhile
  \State \Return{$time[|N|-1]$}
  \EndFunction
\end{algorithmic}

\begin{framed}
  \noindent
  \textbf{Computational cost}. See the previous section.
\end{framed}

\subsection{Solution 3: BFS-like traversal}

\begin{algorithmic}[1]
  \Function{reach}{$N$, $M$, $A[]$, $B[]$, $appear[]$, $disappear[]$}
      \For{$i=0$ to $M-1$}
          \State $edges\_[A[i]].push\_back(make\_pair(i, B[i]))$
          \State $edges\_[B[i]].push\_back(make\_pair(i, A[i]))$
      \EndFor
      \For{$i=0$ to $N-1$}
          \State $done\_[i] \gets false$
          \State $distance\_[i] \gets \infty$
      \EndFor
      \State $reached\_[0].push\_back(0)$
    \State $distance\_[0] \gets 0$
    \For{$t=0$ to $MAX\_TIME$}
      \ForAll{$v \in reached\_[t]$}
          \If{not $done\_[v]$}
          \ForAll{$edge \in edges\_[v]$}
            \State $staircase \gets edge.first$
            \State $neighbor \gets edge.second$
            \State $time \gets \max(distance_[v], appear[staircase])+1$
            \If{not $done\_[neighbor]$ \\ \hfill and $distance\_[v] < disappear[staircase]$ \\ \hfill and  $time < distance\_[neighbor]$}
              \State $distance\_[neighbor] \gets time$
              \State $reached\_[time].push\_back(neighbor)$
            \EndIf
          \EndFor
        \State $done\_[v] \gets true$
          \EndIf
      \EndFor
    \EndFor
    \State \Return{$(distance\_[N-1] = \infty) ? -1 : distance\_[N-1]$}
  \EndFunction
\end{algorithmic}

\begin{framed}
  \noindent
  \textbf{Computational cost}: $O(m + MAX\_TIME)$.
\end{framed}


\section{Paletta}

Paletta ordering\footnote{\url{http://didawiki.cli.di.unipi.it/lib/exe/fetch.php/magistraleinformatica/alg2/algo2_16/paletta.pdf}}
is a peculiar ordering technique: given a 3-tuple of elements, paletta takes the
central element as pivot and swaps the two elements right before and next to it.
To make an example:
$$(3, 2, 1) \xrightarrow{paletta} (1, 2, 3)$$
We now want to develop an algorithm to order any array through paletta ordering
with the minimal number of swaps.
You should see as not every array can be ordered (e.g. [1, 3, 2]).

\subsection{Solution 1: Split and count-inversions}

We should note that the following properties hold:
\begin{enumerate}
    \item Every element can be a pivot, but the first and the last one, as they
    have respectively no elements before and after them.
    \item Every element can be swapped as many times as necessary, but only with
    elements of the same 2-remainder (numbers in even positions can only be
    swapped with numbers in even positions, the same holds for odd indexes).
    More formally, if $n$ is the size of the array $A$ we want to sort,
    $i,j \in [1, n - 2]$, $A[i]$ can be swapped with $A[j]$ if and only if $i \equiv j \pmod{2}$.
    \item The least number of swaps does not backtrack any element.
    Formally, let \emph{k} be the minimal number of swaps applied to an array,
    backtracks included. By hypothesis, \emph{k} is minimal, but at least \emph{m},
    $m > 0$ backtrack swaps have been operated, therefore we found a
    $k' = k - m: k' < k$, a new minimal number of swaps: contradiction.
\end{enumerate}

Given item 2, we can split our array in two, even and odd numbers, and order
them counting the swaps.
In our example we'll use \emph{mergesort}, as it runs in $O(n\log n)$, does
backtrack elements, and is very well-known.
Clearly, given an array, a swap happens when an element is pushed back, pulling
the one between its new position and the old one ahead: we can map this behaviour
in the merge routine of mergesort: the array merged is able to push back elements
from its right pointer to the new array, moving them back of $(m - i) + (j - m)$ positions,
where \emph{m} is the dimension of the current two sub-arrays to merge.
Provided that our edited version of mergesort ran successfully on both the
even-index and odd-index, we now need to verify if by merging them we obtain an
ordered array.
Intuitively, the merged array will start with the first element of the even-index
arrays, followed by the first of the odd-index array, followed by the second of
the even-index array, and so on.
To check for these elements is pretty trivial and can be done in linear time.
Follows the pseudo-code for the edited version and \textsc{snake\_check} function:

\begin{algorithmic}[1]
  \Function{merge\_with\_paletta}{$left$, $right$, $k$}:
    \State \dots                        \Comment{merge instructions}
    \If{$right > left$}
      \State $paletta\_count \gets paletta\_count + 1$
      \State \dots
      \EndIf
    \EndFunction
\end{algorithmic}

\begin{algorithmic}[1]
  \Function{snake\_check}{}
    \State $even, odd \gets 0$
    \For{;$even, odd < N;even = even + 1, odd = odd + 1$}
      \If{$a[even] > a[odd]$}
        \State \Return{$-1$}\;
      \EndIf
    \EndFor
    \State \Return{$paletta\_count$}\;
    \EndFunction
\end{algorithmic}

\begin{framed}
  \noindent
  \textbf{Computational cost}: $\Theta(n\log n)$.
\end{framed}
