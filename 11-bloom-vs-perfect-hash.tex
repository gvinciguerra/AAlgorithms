\section{Bloom filters vs. space-efficient perfect hash}

Recall that classic Bloom filters use roughly $1.44\log_2(1/f)$ bits per key, as seen in class (where $f=(1-p)^k$ is the failure probability minimized for $p \approx e^{-\frac{kn}{m}} = 1/2$). The problem asks to extend the implementation required in Problem 10 by employing an additional random universal hash function $s : U \to [m]$ with $m = \lceil 1/f \rceil$, called signature, so that $s(x)$ is also stored (in place of $x$, which is discarded). The resulting space-efficient perfect hash table $T$ has now a one-side error with failure probability of roughly $f$, as in Bloom filters: say why. Design a space-efficient efficient implementation of $T$, and compare the number of bits per key required by $T$ with that required by Bloom filters.

\vspace{1cm}
\noindent
\textbf{Solution.} 