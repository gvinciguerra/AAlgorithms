\section{Bloom filters vs.\ space-efficient perfect hash}

Recall that classic Bloom filters use roughly $1.44\log_2(1/f)$ bits per key, as
seen in class (where $f=(1-p)^k$ is the failure probability minimized for
$p \approx e^{-\frac{kn}{m}} = 1/2$).
The problem asks to extend the implementation required in Problem 10 by employing
an additional random universal hash function $s : U \to [m]$ with $m = \lceil 1/f \rceil$,
called signature, so that $s(x)$ is also stored (in place of $x$, which is discarded).
The resulting space-efficient perfect hash table $T$ has now a one-side error with
failure probability of roughly $f$, as in Bloom filters: say why.
Design a space-efficient efficient implementation of $T$, and compare the number
of bits per key required by $T$ with that required by Bloom filters.

\vspace{0.5cm}
\paragraph{Solution.}
We extend the data structure discussed in Section \ref{section:space-efficient-perfect-hash} with an additional array $C$ that stores the $n$ signatures. Leveraging the fact that $B$ has as many 1s as the number of signatures we need to store, we fill the array $C$ with the signatures in this way: the entry $C[i]$ contains the signature $s(k)$ in binary if and only if $k$ is the key that caused a the $i$th one in $B$. We also need an additional space of $o(X)$ to perform rank-select operations  on $B$ in $O(1)$.

With this new data structure, given a key $l \in U$, to check whether it belongs to $S$ we check whether $B[i]=1$ where $i=\phi(h(l))+h_{h(l)}(l)$ as before, but also:
\begin{equation}
  \label{equation:query-signature}
  C[rank_1(i)-1] \stackrel{?}{=} s(l).
\end{equation}
In fact, $rank_1(i)-1$, executed on $B$, returns the position in $C$ where we find the signature for the key $k\in S$ that caused $B[i]=1$.

We have an error whenever $l\notin S$, but \eqref{equation:query-signature} is true. Since the signature has length $1/f$, $\Pr(error) \leq \frac{1}{1/f} = f$.

\paragraph{Bits per key comparison.} If we consider only the space used by $C$, $B$ and $L$, our solutions uses $n\log_2\frac{1}{f}+X+(X+n)$ bits to store $n$ keys. Since $\E{X}<2n$, on average we need approximately $\log_2\frac{1}{f}+5$ bits per key, that compared to the Bloom filters is better for $f<2^{-\frac{5}{0.44}}$.
