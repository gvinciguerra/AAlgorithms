\section{Count-min sketch: extension to negative counters}

Consider the two-level perfect hash tables presented in [CLRS] and discussed in class.
As already discussed, for a given set of $n$ keys from the universe $U$, a random
universal hash function $h : U \to [m]$ is employed where $m = n$, thus creating
$n$ buckets of size $n_j \geq 0$, where $\Sigma_{j=0}^{n-1}(n_j) = n$.
Each bucket $j$ uses a random universal hash function $h_j: U \to [m]$ with
$m = n_j^2$.
Key $x$ is thus stored in position $h_j(x)$ of the table for bucket $j$, where
$j = h(x)$.
This problem asks to replace each such table by a bitvector of length $n_j^2$,
initialized to all $0$s, where key $x$ is discarded and, in its place, a bit $1$
is set in position $h_j(x)$ (a similar thing was proposed in Problem 4 and thus
we can have a one-side error).
Design a space-efficient implementation of this variation of perfect hash, using
a couple of tips.
First, it can be convenient to represent the value of the table size in unary
(i.e., x zeroes followed by one for size $x$, so $000001$ represents $x = 5$ and
$1$ represents $x = 0$).
Second, it can be useful to employ a rank-select data structure that, given any
bit vector B of b bits, uses additional $o(b)$ bits to support in $O(1)$ time the
following operations on B:
\begin{itemize}
\item $rank_1(i)$: return the number of $1$s appearing in the first $i$ bits of $B$.
\item $select_1(i)$: return the position $i$ of the $j^{th} 1$, if any, appearing
in $B$ (i.e. $B[i] = 1$ and $rank_1(i) = j$).
\end{itemize}

Operations $rank_0(i)$ and $select_0(j)$ can be defined in the same way as above.
Also, note that $o(b)$ stands for any asymptotic cost that is smaller than
\\ $O(b), b \to \infty$.

\subsection{Wide bitvector}

The two-level hashing layers are comprised of:
\begin{enumerate}
\item An hash function and $n$ pointers, each addressing one bitvector $B_j$.
\item $n$ hash functions $h_j$ hashing to their relative buckets $B_j$.
\end{enumerate}

Given that by removing one hash function we would not be able to retrieve any element in its bucket, we try and optimize the pointers on the first level.

\subsubsection{First level pointers}
We merge the bitvectors $B_j$ subsequently one after the other in a flat array $B$: note that the space is not reduced nor increased, as no further bits are used to separate two adjacent buckets.

\begin{equation*}
B = 0 0 1 0 1 1 0 1 \dots 0 1
\end{equation*}

We then store their respective dimensions in unary form in an auxiliary bitvector $L$ following the same principle.

\begin{equation*}
L = 0 0 1 | 0 1 | 1 | 0 1 | \dots 0 1 |
\end{equation*}


It is trivial to see as given a substring $S'$ in $L$ the length it encodes ends at the very next $1$.
We can exploit this property to compute the size of any interval of buckets $B_a$ and $B_b, b \geq a$ in $O(1)$:
\begin{itemize}
\item Pick the $b^{th} 1 = \alpha $ with $select_1(b)$ in $O(1)$: this will give us the end of the precedent bucket.
\item Pick the sum of the sizes of the preceding buckets by computing the number of $0$s with $rank_0(\alpha)$: this will give us starting position of the desired bucket.
\end{itemize}

Given a bitvector $B$ we employ a map structure to answer the $\mathcal{Q}(i)$ in $O(1)$:
\begin{itemize}
\item Hash on the first level in $O(1)$.
\item Given $j = h(k)$, apply the above algorithm in $O(1)$ to select the corresponding bucket starting at index $B[k]$ and ending at $B[k + \delta], \delta \geq 0$.
\item Apply $h_j(k)$ to $B[k, k + \delta]$ and return $B[k + j]$.
\end{itemize}


\subsubsection{Hash functions optimization}
We now try and improve the space for the hash functions parameters $a_j, b_j, p_j$.
First, we select the lowest $p_j$ possible, that is the first $p_j > n_j^2$: by \href{https://en.wikipedia.org/wiki/Bertrand\%27s_postulate}{Bertrand postulate} such a prime $p_j$ exists for $n_j \leq p_j \leq 2n_j - 2$.
We therefore need a space of at most $2n_j^2$ bits for each parameter in an unary encoding.
Since we have $a \in \mathbb{Z}_{p_j}^*, b \in \mathbb{Z}_{p_j}$, we have a space of $3 \cdot 2n_j^2$ for a triple $(a_j, b_j, p_j)$ of any given hash function.
Let us now compute the expected space for the $m$ hash functions:
\begin{equation*}
\E{3 \sum_{j=0}^{m-1}2n_j^2} =
6 \cdot \E{\sum_{j=0}^{m-1}n_j^2}
\end{equation*}
where $\E{n_j^2} \leq 2n$ as shown in the perfect hashing analysis.
Therefore
\begin{equation*}
6 \cdot \E{\sum_{j=0}^{m-1} n_j^2} \leq 6 \cdot 2n = 12n
\end{equation*}