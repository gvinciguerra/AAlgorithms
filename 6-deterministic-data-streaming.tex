\section{Deterministic data streaming}

Consider a stream of n items, where items can appear more than once. The problem is to find the most frequently appearing item in the stream (where ties are broken arbitrarily if more than one item satisfies the latter). For any fixed integer $k \geq 1$, suppose that only $k$ items and counters can be stored, one item per memory cell, where each counter can use only $O(polylog(n))$ bits (i.e. $O(log^c n)$ for any fixed constant $c > 0$): in other words, only $b = O(k\cdot polylog(n))$ bits of space are available. (Note that, even though we call them counters, they can actually contain any kind of information as long as it does not exceed that amount of bits.)
Show that the problem cannot be solved deterministically under the following rules: the algorithm can only use $b$ bits, and read the next item of the stream, one item at a time. You, the adversary, have access to all the stream, and the content of the $b$ bits stored by the algorithm: you cannot change those $b$ bits and the past, namely, the items already read by the algorithm, but you can change the future, namely, the next item to be read. Since the algorithm must be correct for any input, you can use any amount of streams to be fed to the algorithm and as many distinct items as you want. [Hint: it is an adversarial argument based on the fact that, for many streams, there can be a tie on the items.]

\vspace{0.5cm}
\paragraph{Solution.} We, the adversaries, can decide the alphabet $\Sigma$ and the content of the stream $s\in \Sigma^*$. The idea is to create a stream generator that forces any deterministic algorithm to reach a configuration with at least two different streams. In fact, the number of configurations of memory for this class of algorithms is bounded because of $b$ and, by the pigeonhole principle, there exist at least two streams that cause the algorithm to transition the same configuration.

Since there are $2^b$ possible configurations of the memory, we choose to create a class of $2^b$ streams of length $n$. Each stream $s$ has two occurrences of any symbol chosen from an alphabet $\Sigma_s$ of size $n/2$. Therefore, every stream has a tie the frequencies of its elements (until we generate the next one). The alphabets $\Sigma_1, \Sigma_2, \dots \Sigma_{2^b}$ are all subset of $\Sigma$, that satisfies the following condition:
$${|\Sigma| \choose n/2} = 2^b$$
With this condition we can generate $2^b$ streams (from just as many alphabets) such that, for any pair of streams, there exist at least one symbols that appear in one stream but not in the other.

We can apply the pigeonhole principle: suppose that the same configuration is reached in two different executions of the algorithm on two streams with symbols, respectively, from $\Sigma_1$ and $\Sigma_2$. When we will emit as $(n+1)$-th character of the stream the symbol $a \in \Sigma_1$ such that $a \notin \Sigma_2$, and query the algorithm for the most frequent item, we will receive an answer $\mathcal{A}$. The algorithm is deterministic, therefore in the same configuration the answer will be the same for each stream. If $\mathcal{A} \neq a$, then the algorithm gave the wrong answer for the first stream. Otherwise, if $\mathcal{A} = a$, then it's correct for the first stream, but not for the second one, because $a \notin \Sigma_2$.

\begin{table}[h]
  \centering
  \begin{tabular}{l|l|l|l|l|l|l|l|l|l|l|l|l|l|}
    \cline{2-12}
    stream with symbols from $\Sigma_1$ & ... & d & d & e & e & f & f & g & g & h & h \\ \cline{2-12} 
    stream with symbols from $\Sigma_2$ & ... & 4 & 4 & 5 & 5 & 6 & 6 & 7 & 7 & 8 & 8 \\ \cline{2-12} 
  \end{tabular}
\end{table}
