\documentclass{article}
\usepackage[utf8]{inputenc}
\usepackage{hyperref}       % For \url{...}
\usepackage{amssymb}        % Math symbols 
\usepackage{amsmath}        % Various commands like \text{...}
\usepackage{mathtools}      %
\usepackage{algpseudocode}  % Pseudocode printer
\usepackage{listings}       % Source code printer
\usepackage{framed}         % To frame the computational cost

\title{Advanced Algorithms Problems and Solutions}
\author{Mattia Setzu}
\date{October 2016}

\lstset{numbers=left} % Print line numbers in \begin{lstlisting}...

\begin{document}

\maketitle

\tableofcontents
\clearpage

\section{Hogwarts}

The Hogwarts School\footnote{\url{http://didawiki.cli.di.unipi.it/lib/exe/fetch.php/magistraleinformatica/alg2/algo2_16/hogwarts.pdf}} is modeled as a graph $G=(V, E)$ where $V$ is the set of castle's rooms and $E \subseteq V \times V$ is the set of the stairs. Each stair is labelled with the time of appearance and disappearance, and can be walked in both directions, therefore the graph is undirected. The goal is to find, if possible, the minimum amount of time required to go from the first to the last room.

\subsection{Solution 1: Preprocessing-then-Dijkstra}

Dijkstra is able to find the shortest path in a graph with non-negative weights on its edges. Our main idea is to create a Dijkstra compatible graph through a \emph{normalize} function, then apply Dijkstra to it in order to find the shortest path.

\paragraph{Normalize -} Normalizes is a recursive procedure starting from node \emph{v}:
\begin{enumerate}
  \item Collect all neighbors of $v$ at given time.
  \item For each of them, collect the traversal time:
  \begin{equation}
    traversal\_time(v', t) = 1 + \begin{cases}
    t           &   \mbox{if } start[v'] \leq t < end[v']   \\
    start[v']   &   \mbox{if } t < start[v']                \\
    \infty      &   \mbox{if } t \geq end[v']               \\
  \end{cases}
\end{equation}

  \item For each of $v'$ apply $normalize(v', t)$.
\end{enumerate}

Once all neighbors are visited, the algorithm terminates.

\begin{framed}
  \noindent
  \textbf{Computational cost}: $\Theta(n^{2})$ if the vertex set is implemented as an array. $O(|E|+|V|\log |V|)$ with Fibonacci heap.
\end{framed}

\subsection{Solution 2: HogwartsDijkstra}

This solution is a modified version of the Dijkstra’s algorithm. After the initialization, the set $Q$ contains all the vertices that needs to be analyzed, and $V \setminus Q$ contains all the vertices whose minimum amount of time has been determined. In the for loop, the algorithm checks whether the estimation of $time[v]$ can be improved by going through $u$: if this alternative ($alt$) improves the walk from the source to $v$, $time[v]$ is updated with the new estimate. At the end of the function, $time[v]$ is the minimum amount of time to reach $v$ from the source node.

\bigskip

\begin{algorithmic}[1]
  \Function{HogwartsDijkstra}{$G$}:
  \State create vertex set $Q$ of unvisited nodes
  \For{each vertex $v \in V$}      \Comment{initialization}
      \State $time[v] \gets \infty$  \Comment{unknown time from source to v}
      \State add $v$ to $Q$          \Comment{all nodes initially in Q}
  \EndFor
  \State $time[0] \gets 0$ \Comment{time from source to source}
  \While{$Q\ne \emptyset$}
      \State $u \gets x \in Q$ with $\min\{time[x]\}$
      \State remove $u$ from $Q$
      \For{each neighbor $v$ of $u$}:
          \If{$time[u] \leq appear[v]$}       
              \State $alt \gets appear[v] + 1$ \Comment{wait the appearance of the stair}
          \ElsIf{$time[u] < disappear[v]$}       
              \State $alt \gets time[u] + 1$       \Comment{use the stair}
          \Else                        
              \State $alt \gets \infty$            \Comment{the stair has already disappeared}
          \EndIf
          \If{$alt < time[v]$}
              \State $time[v] \gets alt$           \Comment{a quicker path to $v$ has been found}
          \EndIf
      \EndFor
  \EndWhile
  \State \Return{$time[|N|-1]$}
  \EndFunction
\end{algorithmic}

\begin{framed}
  \noindent
  \textbf{Computational cost}. See the previous section.
\end{framed}

\subsection{Solution 3: BFS-like traversal}

\begin{algorithmic}[1]
  \Function{reach}{$N$, $M$, $A[]$, $B[]$, $appear[]$, $disappear[]$}
      \For{$i=0$ to $M-1$}
          \State $edges\_[A[i]].push\_back(make\_pair(i, B[i]))$
          \State $edges\_[B[i]].push\_back(make\_pair(i, A[i]))$
      \EndFor
      \For{$i=0$ to $N-1$}
          \State $done\_[i] \gets false$
          \State $distance\_[i] \gets \infty$
      \EndFor
      \State $reached\_[0].push\_back(0)$
    \State $distance\_[0] \gets 0$
    \For{$t=0$ to $MAX\_TIME$}
      \For{$v \in reached\_[t]$}
          \If{not $done\_[v]$}
          \For{$edge \in edges\_[v]$}
            \State $staircase \gets edge.first$
            \State $neighbor \gets edge.second$
            \State $time \gets max(distance_[v], appear[staircase])+1$
            \If{not $done\_[neighbor]$ \\ \hfill and $distance\_[v] < disappear[staircase]$ \\ \hfill and  $time < distance\_[neighbor]$}
              \State $distance\_[neighbor] \gets time$
              \State $reached\_[time].push\_back(neighbor)$
            \EndIf
          \EndFor
        \State $done\_[v] \gets true$
          \EndIf
      \EndFor
    \EndFor
    \State \Return{$(distance\_[N-1] == \infty) ? -1 : distance\_[N-1]$}
  \EndFunction
\end{algorithmic}

\begin{framed}
  \noindent
  \textbf{Computational cost}: $O(m + MAX\_TIME)$.
\end{framed}

\section{Paletta}
\href{http://didawiki.cli.di.unipi.it/lib/exe/fetch.php/magistraleinformatica/alg2/algo2_16/paletta.pdf}{Problem description}.
\subsection{Solution 1: Split and count-inversions}

\section{Nemesis}
\href{http://didawiki.cli.di.unipi.it/lib/exe/fetch.php/magistraleinformatica/alg2/algo2_16/nemesi.pdf}{Problem description}.
\subsection{Solution 1: Naive}

\subsection{Solution 2: Coloring}

\subsection{Solution 3: Ordered matrix}


\end{document}