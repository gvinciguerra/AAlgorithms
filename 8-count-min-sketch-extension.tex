\section{Count-min sketch: extension to negative counters}

Check the analysis seen in class, and discuss how to allow $F[i]$ to change by arbitrary
values read in the stream.
Namely, the stream is a sequence of pairs of elements, where the first element indicates the
item $i$ whose counter is to be changed, and the second element is the amount $v$ of that
change ($v$ can vary in each pair).
In this way, the operation on the counter becomes $F[i] = F[i] + v$, where the increment
and decrement can be now seen as $(i, 1)$ and $(i, -1)$.


\subsection{First solution}

Let $F$ be the implicit vector of size $n$ that receives updates $(i, v)$ such that $F[i]=F[i]+v$ (initially, $F[i]=0 \; \forall i \in [1, n]$).
The Count-Min (CM) sketch data structure, with parameters $\varepsilon$ and $\delta$, consists of:
\begin{itemize}
  \item a table $T$ of size $r\times c$, where $r=\log\frac{1}{\delta}$ and $c=\frac{e}{\varepsilon}$, that we use to approximate queries on $F$.
  \item $r$ hash functions $h_1, \dots, h_r$, chosen uniformly at random from a pairwise-independent family, that maps from $\{1,\dots,n\}$ to $\{1,\dots,c\}$.
\end{itemize}
The parameters specify that the error in answering a query is in within a factor of $\varepsilon$ with probability $1-\delta$.
When an update $(i, v)$ arrives, the table is updated as follows: 
$$ T[j, h_j(i)] = T[j, h_j(i)] + v \qquad \forall j = 1, \dots, r $$ 
A point query returns an approximation $\tilde{F}[i]$ of $F[i]$. If we assume that $F[i] \geq 0$, then:
$$\tilde{F}[i]=\min_{j\in[1,r]} T[j, h_j(i)]$$ with the guarantees:
\begin{enumerate}
  \item $\tilde{F}[i]\geq F[i]$ 
  \item $\prob{\tilde{F}[i]>F[i]+\varepsilon\norm{F}}\le\delta$.
\end{enumerate}

The problem statement asks to remove the assumption $F[i]\geq 0$. In this case the point query becomes: 
$$\tilde{F}[i]=\median_{j\in[1,r]} T[j, h_j(i)]$$
with the guarantee that $\prob{F[i]-3\varepsilon\norm{F} \leq \tilde{F}[i] \leq F[i]+3\varepsilon\norm{F}} < 1-\delta^{1/4}$.

\begin{proof}
  Let $I_{i,j,k}$ the indicator variable that is $1$ if $i \ne k \wedge h_j(i)=h_j(k)$ and $0$ otherwise. Observe that, by pairwise independence of the hash functions,
  $$\E{I_{i,j,k}} = \prob{h_j(i)=h_j(k)} \leq \frac{1}{c} = \frac{\varepsilon}{e} $$
  and, given $X_{i,j}=\sum_{k=1}^n I_{i,j,k}F[k]$
  \begin{equation}
    \label{cm-sketch-1}
  	\E{|X_{i,j}|} = \E{\sum_{k=1}^n I_{i,j,k}|F[k]|} \leq \sum_{k=1}^n |F[k]|\E{I_{i,j,k}} \leq \frac{\varepsilon}{e}\norm{F}
  \end{equation}
  The probability that any count (estimate of $F[i]$) is off by more than $3\varepsilon\norm{F}$ is:
  \begin{align*}
    &  \prob{|T[j, h_j(i)]-F[i]| \geq 3\varepsilon\norm{F}} &  \\
    &= \prob{|F[i]+X_{i,j}-F[i]| \geq 3\varepsilon\norm{F}} & \text{by construction of $T$} \\
    &= \prob{|X_{i,j}| \geq 3\varepsilon\norm{F}} & \\
    &\leq \frac{\E{|X_{i,j}|}}{3\varepsilon\norm{F}} & \text{by the Markov inequality} \\
    &\leq \frac{\varepsilon\norm{F}}{e} \frac{1}{3\varepsilon\norm{F}} & \text{by \eqref{cm-sketch-1}} \\
    &= \frac{1}{3e} < \frac{1}{8} &
  \end{align*}
  
  The median is a bad choice if more than half of the $r = \ln 1/\delta$ estimates are bad. We define $Y$ to be the number of bad estimates: $Y = \sum_{j=1}^r Y_j$, where $Y_j$ is 1 if $|X_{i,j}| \geq 3\varepsilon\norm{F}$ and 0 otherwise. Hence the median is a bad choice if the sum $Y$ is at least $r/2$, and this happens with probability:
  $$\prob{Y > \frac{1}{2}\ln\frac{1}{\delta}}$$
  We can apply the Theorem \ref{theorem:chernoff} (with $\mu = \E{Y} = \sum_{j=1}^r \prob{j \text{ is a bad estimate}} < r/8$) to upper bound this probability. 
\end{proof}

\subsection{Second solution}

We trivially have some changes over $X_{ji}$ and $\tilde{F}[i]$:
    \begin{gather*}
        X_{ji} = \Sigma^{n}(I_k F[k]) \\
        \tilde{F}[i] = F[i] + X_{ji}
    \end{gather*}
\label{X_ji}\paragraph{$X_{ji}$} can vary as complementary increments $(+v_i, +v_j, +v_k, -v_k, -v_j, -v_i)$
can make it $\leq 0$ without necessarily being updates on $i$.
In order to have a minimum error estimate we'll pick the \emph{median} value of $X_{ji}$.
\label{f_tilde}~\paragraph{$\tilde{F}[i] = F[i] + X_{ji}$} by the above assertion the counter $\tilde{F}[i]$ could have no garbage, or negative
garbage according to the other increments.
Therefore by choosing the minimum $\tilde{F}[i]$ over the $\log(\frac{1}{\delta})$ we
could actually pick the most perturbed result (think of a collision with the
biggest negative increment over one row).

\label{probability}The last step is the probability proof: $\prob{\tilde{F}[i] > F[i] + \epsilon \norm{F}}$.
Since by~\ref{X_ji} $X_{ji}$ can be either positive or negative we pick the \emph{median}
value $med$ over the minimum $m$ and maximum $M$.
We are then able to split our error analysis in two of equivalent size $\frac{r}{2}$:
\begin{gather*}
\prob{\tilde{F}[i] \in F[i] + \epsilon \norm{F}} = \\
\prob{\tilde{F}[i] \geq F[i] - \abs{\epsilon} \norm{F} \wedge \tilde{F}[i] \leq F[i] + \abs{\epsilon} \norm{F}} = \\
\prob{\tilde{F}[i] \geq F[i] - \abs{\epsilon} \norm{F}) \prob(\tilde{F}[i] \leq F[i] + \abs{\epsilon} \norm{F}}
\end{gather*}
Let us define $p_0 = \prob{\tilde{F}[i] \geq F[i] - \abs{\epsilon} \norm{F}}$ and
$p_1 = \prob{\tilde{F}[i] \leq F[i] + \abs{\epsilon} \norm{F}}$.
We can report the analysis seen in class adjusted for the number of rows $\frac{r}{2}$

\begin{align*}
p_0 = \prod^{\frac{r}{2}}\prob{|Xji| \geq - \abs{\epsilon} \norm{F}} = \\
    = - \frac{1}{2^{\frac{r}{2}}} = - \delta^{\frac{1}{2}}
\end{align*}

Now for $p_1$:

\begin{align*}
    p_1 = \prob{\tilde{F}[i] \leq F[i] + \abs{\epsilon} \norm{F}} = \\
    1 - \prob{\tilde{F}[i] \geq F[i] + \abs{\epsilon} \norm{F}} =
    = 1 - \frac{1}{2^{\frac{r}{2}}} = 1 - \delta^{\frac{1}{2}}
\end{align*}

Giving us a combined probability of $p = p_0 p_1 =
(- \delta^{\frac{1}{2}}) (1 - \delta^{\frac{1}{2}}) = -\sqrt{\delta} + \delta$.
