\section{Count-min sketch: extension to negative counters}

Check the analysis seen in class, and discuss how to allow $F[i]$ to change by arbitrary
values read in the stream.
Namely, the stream is a sequence of pairs of elements, where the first element indicates the
item $i$ whose counter is to be changed, and the second element is the amount $v$ of that
change ($v$ can vary in each pair).
In this way, the operation on the counter becomes $F[i] = F[i] + v$, where the increment
and decrement can be now seen as $(i, 1)$ and $(i, -1)$.


\subsection{First solution}

Let $F$ be the implicit vector of size $n$ that receives updates $(i, v)$ such that $F[i]=F[i]+v$ (initially, $F[i]=0 \; \forall i \in [1, n]$).
The Count-Min (CM) sketch data structure, with parameters $\varepsilon$ and $\delta$, consists of:
\begin{itemize}
  \item a table $T$ of size $r\times c$, where $r=\log\frac{1}{\delta}$ and $c=\frac{e}{\varepsilon}$, that we use to approximate queries on $F$.
  \item $r$ hash functions $h_1, \dots, h_r$ chosen uniformly at random from a pairwise-independent family. 
\end{itemize}
When an update $(i, v)$ arrives, the table is updated as follows: 
$$ T[i, h_j(i)] = T[i, h_j(i)] + v \qquad \forall j = 1, \dots, r $$ 
A point query returns an approximation $\tilde{F}[i]$ of $F[i]$. If we assume that $F[i] \geq 0$, then:
$$\tilde{F}[i]=\min_{j\in[1,r]} T[j, h_j(i)]$$ with the guarantees:
\begin{enumerate}
  \item $\tilde{F}[i]\geq F[i]$ 
  \item $\prob(\tilde{F}[i]>F[i]+\varepsilon\norm{F})\le\delta$.
\end{enumerate}

The problem statement asks to remove the assumption $F[i]\geq 0$. In this case the point query becomes: 
$$\tilde{F}[i]=\median_{j\in[1,r]} T[j, h_j(i)]$$
with the guarantee that $\prob(F[i]-3\varepsilon\norm{F} \leq \tilde{F}[i] \leq F[i]+3\varepsilon\norm{F}) < 1-\delta^{1/4}$.

% TODO: Proof

\subsection{Second solution}

We trivially have some changes over $X_{ji}$ and $\tilde{F}[i]$:
    \begin{gather*}
        X_{ji} = \Sigma^{n}(I_k F[k]) \\
        \tilde{F}[i] = F[i] + X_{ji}
    \end{gather*}
\label{X_ji}\paragraph{$X_{ji}$} can vary as complementary increments $(+v_i, +v_j, +v_k, -v_k, -v_j, -v_i)$
can make it $\leq 0$ without necessarily being updates on $i$.
In order to have a minimum error estimate we'll pick the \emph{median} value of $X_{ji}$.
\label{f_tilde}~\paragraph{$\tilde{F}[i] = F[i] + X_{ji}$} by the above assertion the counter $\tilde{F}[i]$ could have no garbage, or negative
garbage according to the other increments.
Therefore by choosing the minimum $\tilde{F}[i]$ over the $\log(\frac{1}{\delta})$ we
could actually pick the most perturbed result (think of a collision with the
biggest negative increment over one row).

\label{probability}The last step is the probability proof: $\prob{\tilde{F}[i] > F[i] + \epsilon \norm{F}}$.
Since by~\ref{X_ji} $X_{ji}$ can be either positive or negative we pick the \emph{median}
value $med$ over the minimum $m$ and maximum $M$.
We are then able to split our error analysis in two of equivalent size $\frac{r}{2}$:
\begin{gather*}
\prob(\tilde{F}[i] \in F[i] + \epsilon \norm{F}) = \\
\prob(\tilde{F}[i] \geq F[i] - \abs{\epsilon} \norm{F} \wedge \tilde{F}[i] \leq F[i] + \abs{\epsilon} \norm{F}) = \\
\prob(\tilde{F}[i] \geq F[i] - \abs{\epsilon} \norm{F}) \prob(\tilde{F}[i] \leq F[i] + \abs{\epsilon} \norm{F})
\end{gather*}
Let us define $p_0 = \prob(\tilde{F}[i] \geq F[i] - \abs{\epsilon} \norm{F})$ and
$p_1 = \prob(\tilde{F}[i] \leq F[i] + \abs{\epsilon} \norm{F})$.
We can report the analysis seen in class adjusted for the number of rows $\frac{r}{2}$

\begin{align*}
p_0 = \prod^{\frac{r}{2}}(\prob(|Xji| \geq - \abs{\epsilon} \norm{F})) = \\
    = - \frac{1}{2^{\frac{r}{2}}} = - \delta^{\frac{1}{2}}
\end{align*}

Now for $p_1$:

\begin{align*}
    p_1 = \prob(\tilde{F}[i] \leq F[i] + \abs{\epsilon} \norm{F}) = \\
    1 - \prob(\tilde{F}[i] \geq F[i] + \abs{\epsilon} \norm{F}) =
    = 1 - \frac{1}{2^{\frac{r}{2}}} = 1 - \delta^{\frac{1}{2}}
\end{align*}

Giving us a combined probability of $p = p_0 p_1 =
(- \delta^{\frac{1}{2}}) (1 - \delta^{\frac{1}{2}}) = -\sqrt{\delta} + \delta$.
