\section{Depth of a node in a random search tree}

A random search tree for a set S can be defined as follows: if $S$ is empty, then
the null tree is a random search tree; otherwise, choose uniformly at random a key
$k$ as root, and the random search trees on $L = \{x \in S : x < k\}$ and $R = \{x \in S :
x > k\}$ become, respectively, the left and right subtree of the root $k$.
Consider the randomized QuickSort discussed in class and analyzed with indicator
variables \href{http://didawiki.cli.di.unipi.it/lib/exe/fetch.php/magistraleinformatica/alg2/algo2_13/randqs.pdf}{CLRS 7.3},
and observe that the random selection of the pivots follows the above process,
with indicator variables, prove that:
\begin{enumerate}
  \item the expected depth of a node (i.e.\ the random variable representing the distance of the node from the root) is nearly $2 \ln n$;
  \item the expected size of its subtree is nearly $2 \ln n$ too, observing that it is a simple variation of the previous analysis;
  \item the that the probability that the depth of a node exceeds $c2 \ln n$ is small for any given constant $c > 1$.
\end{enumerate}

\subsection{Proof with indicator variable}

\paragraph{Prove that the expected depth of a node is nearly $2 \ln n$.}

\begin{proof}
  Let $z_m$ the $m$th smallest element in $S$ and $$X_{ij}=
  \left\{
    \begin{array}{ll}
      1 & \mbox{if $z_j$ is an ancestor of $z_i$ in the random search tree} \\
      0 & \mbox{otherwise}
    \end{array}
  \right.$$
  The depth of the node $i$ in the tree is given by the number of its ancestors:
  \begin{equation}
    X=\sum_{\substack{j=1 \\ j\ne i}}^n X_{ij}
    \label{equation:node-depth}
  \end{equation}

  Note that the depth of a node is also equal to the number of comparison it's
  involved in (in other words, the number of times it became the left or the
  right child of a randomly chosen pivot).

  Once a pivot $k$ is chosen from $S$, $S$ is partitioned in two subsets $L$ and
  $R$. The elements in the set $L$ will not be compared with the elements in $R$
  at any subsequent time. The event $E_1=$ ``$z_j$ is an ancestor of $z_i$ in the
  random search tree'' occurs if $z_j$ and $z_i$ belong to the same partition
  \emph{and} $z_j$ was chosen as pivot before $z_i$. The probability that $E_1$
  occurs, since it is the intersection of two events, can be upper bounded by:
  $$\text{Pr}\{ z_j \text{ was chosen as pivot before } z_i
  \}=\frac{1}{\text{size of the partition}}\leq\frac{1}{|j-i|+1}$$ because
  pivots are chosen randomly and independently, and because the partition that
  contains both $z_j$ and $z_i$ must contain \emph{at least} the $|j-i|+1$
  numbers between $z_j$ and $z_i$.

  Taking expectations of both sides of~\eqref{equation:node-depth}, and then
  using linearity of expectation, we have:
  \begin{align*}
    \E{X} & = \sum_{\substack{j=1\\ j\ne i}}^n \E{X_{ij}} \\
    & = \sum_{\substack{j=1\\ j\ne i}}^n \text{Pr}\{ z_j \text{ is an ancestor
      of } z_i \text{ in the random search tree} \} \\
    & \le \sum_{\substack{j=1\\ j\ne i}}^n \frac{1}{|j-i|} \\
    & = \sum_{j=1}^{i-1} \frac{1}{i-j} + \sum_{j=i+1}^{n} \frac{1}{j-i}
  \end{align*}
  With the change of variables $l=i-j$ and $m=j-i$:
  \begin{equation*}
    = \sum_{l=1}^{i-1} \frac{1}{l} + \sum_{m=1}^{n} \frac{1}{m} \approx 2\ln n
  \end{equation*}

\end{proof}

\paragraph{Prove that the expected size of its subtree is nearly $2 \ln n$ too,
observing that it is a simple variation of the previous analysis.}

\begin{proof}
  The size of the subtree of a randomly chosen pivot of $z_j\in S$ is given by
  the number of it's descendants. Since~\eqref{equation:node-depth} is the
  number of ancestors of a node $z_i$, we can find the number of descendants of
  $z_j$ by changing the summation from $j=1,\dotsc,n$ to $i=1,\dotsc,n$.
\end{proof}

\paragraph{Prove that the that the probability that the depth of a node exceeds $c2 \ln n$ is small for any given constant $c > 1$.}

\begin{proof}
  We apply the Theorem \ref{theorem:chernoff} below with $(1+\delta)=c$ and $\mu=2 \ln n$:
  $$\prob{X>2c\ln n}<\left(\frac{e^{c-1}}{c^c}\right)^{2 \ln n}$$
  Since $\lim_{n\to+\infty}a^n=0$ with $0<a<1$, and $\lim_{n\to+\infty}2\ln n= +\infty$, we have to prove that, for any $c>1$
  $$0<\frac{e^{c-1}}{c^c}<1$$
  The first inequality is simple to verify, for the latter observe that
  $$\frac{e^{c-1}}{c^c}<1 \iff e^{c-1}<c^c \iff c-1 < c\ln c \iff c(1-\ln c)<1.$$
\end{proof}

\newtheorem{thm}{Theorem}
\begin{thm}[Chernoff Bound]\label{theorem:chernoff}
  Let $X_1, \dots, X_n$ be independent Poisson trials such that, for $1 \leq i \leq n$, $\prob{X_i=1}=p_i$, where $0 < p_i < 1$. Then, for $X=\sum_{i=1}^n X_i$, $\mu=\E{X}=\sum_{i=1}^n p_i$ and any $\delta>0$,
	$$\prob{X>(1+\delta )\mu}<\left({\frac {e^{\delta }}{(1+\delta )^{(1+\delta )}}}\right)^\mu.$$
\end{thm}

\subsection{Recursive balanced proof}

Let $n$ be the number of nodes in the input list $l$, $h = \log_2(n)$ the height
of a balanced tree over $l$, $T(p)$ the tree built over the permutation $p$ of pivots,
$d(m)$ be the positional distance of a value $m$ of a partition from the median
value of the said partition.
Then the following holds:

    \begin{itemize}
    \item $height(T) = h \iff |T.left| = |T.right| \pm 1$ Trivially, let $r$ be
    the root  of a 3-nodes partition: then, if the partition is unbalanced, the
    lesser one will comprise of 0 nodes, while the greater one of 2, which implies
    that $height(T.right) == 2$.
    \label{k_distance} \item P = pivot,
    $d(m) = \pm k \implies height(T.left) = height(T.right) \pm k$.
    Recursively from the previous statement, a partition unbalanced of one element
    generates subtrees whose levels differ on a factor of $1$.
    By iterating recursively, their subtrees, if unbalanced by $1$, will yield
    one more level difference.
    Over $k$ unbalanced pivots on a single subtree, at most $k$ levels will be
    added to $h$.
    \item By the previous statement, it follows that $\nexists T, T': height(T) >= height(T')$,
    T balanced, T' unbalanced.
    As stated, let $T', T$ be the unbalanced/balanced tree respectively; let us
    cheat with $T$ and switch the root pivot with the first element in its subtree.
    Now, let us prove by contradiction that $T$ can't stay balanced and that its
    height will increase.
    By shifting the tree to the left we have deprived $T.right$ of either 0 levels
    (in case $T.right$ is able to switch every pivot in its tree with its right
    subtree root, ending with the rightmost leaf in its subtree) or 1, in case no
    rightmost leaf is present.
    Therefore $height(T) <= height(T')$.
    \item The completely unbalanced tree is the tree with the most levels.
    By taking partitions of size 0 we costantly force, at each level, one subtree
    to disappear.
    Therefore, its level(s) has to be necessarly transferred to its brother.
    We then have exactly one node per level, therefore $n$ levels.
    \end{itemize}

\subparagraph{Behaviour on random permutations}
Now let us analyse how the tree depth varies according to random pivot selection.
We start by applying the~\ref{k_distance}k-distance to a tree $T$ with $n = 3$
nodes.
Trivially, $height(T)$ with balanced tree is equal to two.
Now, let us pick either the lowest or the greatest pivot possible: the tree
is unbalanced towards either the left or the right, but $height(T) = 2$ in
both cases.
As the reader can see from~\ref{k_distance}, the distance works in absolute
value; it is then clear how, at every permutation for a pivot $p$, out of the
$n$, there are $2$ that generate a tree of the same height:
$p = d(P) + k, p = d(P) - k$.
Given that at every iteration a node $x$ in a completely unbalanced tree $T$'
has a probability of $\frac{1}{n - i}$, we can define the probability of $x$
being a pivot at level $l$ as:
    \begin{equation}
    P(x_{k}) = \frac{1}{n - l}
    \end{equation}
Now, in order for $x$ not to be chosen as pivot in the previous $l - 1$
levels we have:
    \begin{equation}
    P(x_{k}) = \Sigma_{k = 1}^{l - 1} (\frac{1}{n - l + 1})
    \end{equation}
Given the height of $T$, the (harmonic) partial series converges to
$\ln{(n)} + 1$.
Let us now add a root $r$ s.t. $T'.right = T, T'.left = T$.
We now have to consider the mirror case $\ln{(n')} + \ln{(n')}$,
given by the previous $n' = n/2$ in the logarithm, since the number of nodes
doubled, the $+1$ removed for both, since now neither of $T'.left, T.right$
is the root, and a $+1$ added since a new level has been added.

\paragraph{Upper bound.}
By hypothesis,
    \begin{equation}
    \E{d(x) > 2 c \ln(n)} <<< 1
    \end{equation}
By definition the ancestor of a node $i$ are indipendent random variables,
and we can apply the \href{https://en.wikipedia.org/wiki/Chernoff_bound}{Chernoff
bounds} over the set ${x: d(x) >= 2 \ln(n)}$ of random variables determining
the expected distance of nodes.
\begin{equation*}
    \prob{X \geq c \E{X}} < e^{-c \ln(\frac{c}{e})\E{X}}
\end{equation*}
Let us consider $X = 1 \forall i == \ln(n)$, the expected depth of $\ln(n)$,
then
\begin{equation*}
    \prob{X \geq c \ln(n)} < e^{-c \ln(\frac{c}{e})\ln(n)}
\end{equation*}